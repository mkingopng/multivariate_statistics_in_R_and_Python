%! Author = noone
%! Date = 22/10/22

% Preamble
\documentclass[11pt]{article}

% Packages
\usepackage{amsmath}

% Document
\begin{document}

\subsection{Course description}\label{subsec:course-description}
This course will give you a solid methodological background in multivariate
analysis as a backbone of applied statistics.
You will learn the theoretical foundations of the most commonly applied
multivariate techniques such as mean vector and covariance matrix estimation
and testing, estimation and testing of correlations, multivariate linear models,
discriminant analysis, classification and support vector machines, principal
components, canonical correlations analysis, cluster analysis, factor analysis
and structural equations.
You will study the properties and the importance of the multivariate normality
assumption in the context of each of these methods.
R-based computing will feature prominently in the course.
At the end of the course you should be able to use all the above techniques in
your work as an applied statistician, for practical analysis of real datasets.

\subsection{Course Learning Outcomes}\label{subsec:course-learning-outcomes}
\begin{enumerate}
  \item Use the general terminology, notation and concepts in the theory, methods and  applications of multivariate analysis.
  \item Use the properties of the multivariate normal distribution to justify statistical inference procedures based on multivariate normality.
  \item Formulate and solve inference problems that use one or a combination of multivariate statistical procedures.
  \item Perform multivariate inference procedures by using the R suite.
  \item Write simple R programs for data input and output and implement simple multivariate analyses by using R.
  \item Apply the multivariate techniques and models to the analysis of datasets, interpret the results and draw conclusions.
\end{enumerate}

\subsection{Assessment}\label{subsec:assessment}
All assessments in this course will be submitted via Moodle.
The project assessment brief and files are stored in Ed, but submission is in Moodle via Turnitin.
Please refer to the Assessments page in Moodle to read more details about the assessment structure in this course.

\begin{enumerate}
  \item Assessment 1: Quiz (10%). A multiple-choice/short answer quiz and/or coding challenge. This assessment assesses your understanding of the course contents covered up to and including Week 1.
  \item Assessment 2: Quiz (10%). A multiple-choice/short answer quiz and/or coding challenge. This assessment assesses your understanding of the course contents covered up to and including Week 2.
  \item Assessment 3: Quiz (10%). A multiple-choice/short answer quiz and/or coding challenge. This assessment assesses your understanding of the course contents covered up to and including Week 3.
  \item Assessment 4: Quiz (10%). A multiple-choice/short answer quiz and/or coding challenge. This assessment assesses your understanding of the course contents covered up to and including Week 4.
  \item Assessment 5: Quiz (10%). A multiple-choice/short answer quiz and/or coding challenge. This assessment assesses your understanding of the course contents covered up to and including Week 5.
  \item Assessment 6: Quiz (10%). A multiple-choice/short answer quiz and/or coding challenge. This assessment assesses your understanding of the course contents covered up to and including Week 6.
  \item Assessment 7: Data analysis report (40%). This is an individual project assessment to be completed progressively and submitted by Monday of Week 7. Apply your knowledge and skills to various data analysis problems.
\end{enumerate}

\subsection{FAQ and Old Webinars}\label{subsec:faq-and-old-webinars}
For your convenience, we have provided an FAQ for each week, comprising last year's webinars (with time indexes) and answers to questions posted to the forum.
Taking a quick look at those will often be a more efficient way to get clarification than asking, since I may not be able to reply immediately.

\subsection{Collaborate sessions (webinar/consultation)}\label{subsec:collaborate-sessions-(webinar/consultation)}

Each week of this course includes online webinar/consultation sessions (Wednesdays 17:00–19:00) in which the Online Course Convenor discusses the weekly material, upcoming assessments, common errors in recent assessments, and other matters depending on student interest, including addressing questions asked in the discussion forum.
The webinars will be recorded and indexed for topics, so that you can go the specific time instance if you wish to review only a specific topic in the recording. (Last year's webinars are also available and indexed.)

If you are unable to attend a webinar, it's highly recommended that you watch a recording of the session afterwards.

To access the recordings, you need to
\begin{enumerate}
  \item login to the Moodle page,
  \item click on ``Communication'' in the left tab,
  \item click on``Launch Collaborate'',
  \item (select/click ``Open in new window'')
  \item once in ``Collaborate'', click the Menu button (three horizontal bars in the top left corner) and select "Recordings".
\end{enumerate}

\subsection{Forum QA}\label{subsec:forum-qa}
This section describes the forum interaction policy for this course, designed to accommodate the nature of this course.
Note that it may be different from other UNSW Online courses you have taken.
Please read it carefully.
A few days before each online webinar session, the instructor will create a QA thread on the discussion forum outlining the planned content of the upcoming webinar.
Please post items that you would like the instructor to cover in that webinar to that QA thread.
These can include questions and requests to clarify materials, to derive or prove a particular proposition, go over a code challenge or solution, or discuss a topic relevant to the course.  (Short questions may be answered in the thread.)
During the webinar, the instructor will (in this order):
Go over the planned material.
Answer questions posed in the QA thread.
Time permitting, take any further questions, as in consultation.
After the session, the instructor will post a timestamped index of planned material and questions on the QA thread.
Note that for questions about the course materials, the instructor will only field those asked in a QA thread.
Questions about course materials asked outside the QA thread may be answered by your peers but may be disregarded by the instructor.
This does not apply to questions about logistics, technical issues (e.g., "Blackboard isn't working"), and similar matters, which will be answered on the forum.
Please tag them appropriately when you post.

\subsection{Optional readings}\label{subsec:optional-readings}
All readings are available from the course Leganto reading list.
Please keep in mind that you will need to be logged into Moodle to access the Leganto reading list.

\subsection{Online Course Convenor and developing academic}\label{subsec:online-course-convenor-and-developing-academic}
Dr Pavel Krivitsky is a Senior Lecturer in Statistics at UNSW Sydney, where he moved in 2019 after 6 years lecturing at University of Wollongong.
He received his PhD in Statistics from University of Washington in 2009 and worked as a postdoctoral research associate at Carnegie Mellon University and Penn State University.
His research interests lie in development of methods and models for analysing complex network data and processes, with applications in epidemiology and the social sciences.
His work has a strong computational component, and he develops and maintains a number of popular R packages on the Comprehensive R Archive Network (CRAN).

  insert video

\subsection{Course Description}
This course will give you a solid methodological background in multivariate
analysis as a backbone of applied statistics.
you will learn the theoretical foundations of the most commonly applied
multivariate techniques such as:

\begin{description}
  \item mean vector and covariance matrix estimation and testing,
  \item estimation and testing of correlations,
  \item multivariate linear models,
  \item discriminate analysis,
  \item classification and
  \item support vector machines
  \item principal components,
  \item canonical correlations analysis,
  \item analysis, factor analysis
  \item and structural equations
\end{description}

You will study the properties and the importance of the multivariate normality
assumption in the context of each of these methods.
R-based computing will feature prominently in the course.
At the end of the course you could be able to use all the above techniques in
your work as an applied statistician for practical analysis of real datasets.

\subsection{Course Learning Outcomes}\label{subsec:course-learning-outcomes2}
\begin{description}
  \item use the general terminoology, notation and concepts in the theory, methods and applications of multivariate analysis
  \item use the properties of the multivariate normal distribution to justify statistical inference procedures based on multivariate normality
  \item formulate and solve inference problems that use one or a combination of multivariate statistical proceedures
  \item perform multivariate inference procedures by using the R suite
  \item write simple R programs for data input and output and implement simple multivariate analysis using RecordingsApply the multivaiate techniques and models to the analyss of datasets  interpret the results and draw conclusions
\end{description}

\subsection{Assessment}\label{subsec:assessment2}
all assessments in this course will be submitted via Moodle.
The project assessment brief and files are stored in Ed but submission is in Moodle via Turnitin.
Please refer to the assessments page in Moodle to read more details about the assessment structure in this course.
\begin{description}
  \item \textbf{assessment 1 quiz 1} (10) A multiple choice short answer quiz and/or coding challenge. This assessment  assesses your understanding of the course contents covered up to and includign week 1
  \item \textbf{assessment 2 quiz 2} (10) A multiple-choice/short answer quiz and/or coding challenge. This assessment assesses your understanding of the course contents covered up to and including Week 2.
  \item \textbf{assessment 3 quiz 3} (10) A multiple-choice/short answer quiz and/or coding challenge. This assessment assesses your understanding of the course contents covered up to and including Week 3.
  \item \textbf{assessment 4 quiz 4} (10) A multiple-choice/short answer quiz and/or coding challenge. This assessment assesses your understanding of the course contents covered up to and including Week 4.
  \item \textbf{assessment 5 quiz 5} (10) A multiple-choice/short answer quiz and/or coding challenge. This assessment assesses your understanding of the course contents covered up to and including Week 5.
  \item \textbf{assessment 6 quiz 6} (10) A multiple-choice/short answer quiz and/or coding challenge. This assessment assesses your understanding of the course contents covered up to and including Week 6.
  \item \textbf{assessment 7 data analysis report} (40) This is an individual project assessment to be completed progressively and submitted by Monday of Week 7. Apply your knowledge and skills to various data analysis problems.
\item \end{description}

\subsection{FAQ and Old Webinars}
For your convenience, we have provided an FAQ for each week, comprising last year's webinars (with time indexes) and answers to questions posted to the forum.
Taking a quick look at those will often be a more efficient way to get clarification than asking, since I may not be able to reply immediately.

\subsection{collaborate sessions (webinar/consultation)}
Each week of this course includes online webinar/consultation sessions
(Wednesdays 17:00–19:00) in which the Online Course Convenor discusses the
weekly material, upcoming assessments, common errors in recent assessments,
and other matters depending on student interest, including addressing questions
asked in the discussion forum.

The webinars will be recorded and indexed for topics, so that you can go the
specific time instance if you wish to review only a specific topic in the
recording. (Last year's webinars are also available and indexed.)

If you are unable to attend a webinar, it's highly recommended that you watch a
recording of the session afterwards.
To access the recordings, you need to
\begin{description}
    \item login to the Moodle page,
    \item click on ``Communication'' in the left tab,
    \item click on ``Launch Collaborate'',
    \\item (select/click ``Open in new window'')
    \item once in ``Collaborate', click the Menu button (three horizontal bars in the top left corner) and select ``Recordings''.
\end{description}

Forum QA

This section describes the forum interaction policy for this course, designed to accommodate the nature of this course. Note that it may be different from other UNSW Online courses you have taken. Please read it carefully.

    A few days before each online webinar session, the instructor will create a QA thread on the discussion forum outlining the planned content of the upcoming webinar.

    Please post items that you would like the instructor to cover in that webinar to that QA thread. These can include questions and requests to clarify materials, to derive or prove a particular proposition, go over a code challenge or solution, or discuss a topic relevant to the course.  (Short questions may be answered in the thread.)

    During the webinar, the instructor will (in this order):

        Go over the planned material.

        Answer questions posed in the QA thread.

        Time permitting, take any further questions, as in consultation.

After the session, the instructor will post a timestamped index of planned
material and questions on the QA thread.

Note that for questions about the course materials, the instructor will only
field those asked in a QA thread.
Questions about course materials asked outside the QA thread may be answered
by your peers but may be disregarded by the instructor.

This does not apply to questions about logistics, technical issues (e.g.,
"Blackboard isn't working"), and similar matters, which will be answered on
the forum.
Please tag them appropriately when you post.

\subsection{Optional readings}
All readings are available from the course Leganto reading list.
\begin{description}
  \item aspects of multivariate statistical theory, ROb Muirhead
  \item applied multivartiate staatistical analysis
  \item applied statistical analysis Hardle, Wolfgang Karl, Simar Leopold, Springer Math
\end{description}
Please keep in mind that you will need to be logged into Moodle to access the
Leganto reading list.
Online Course Convenor and developing academic

Dr Pavel Krivitsky is a Senior Lecturer in Statistics at UNSW Sydney, where he
moved in 2019 after 6 years lecturing at University of Wollongong.
He received his PhD in Statistics from University of Washington in 2009 and
worked as a postdoctoral research associate at Carnegie Mellon University and
Penn State University.
His research interests lie in development of methods and models for analysing
complex network data and processes, with applications in epidemiology and the
social sciences.
His work has a strong computational component, and he develops and maintains a
number of popular R packages on the Comprehensive R Archive Network (CRAN).

\end{document}