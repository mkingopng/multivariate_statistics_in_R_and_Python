%! Author = noone
%! Date = 28/10/22

% Preamble
\documentclass[11pt]{article}

% Packages
\usepackage{amsmath}

% Document
\begin{document}

Dr Pavel Krivitsky gives you a brief overview of topics and concepts we'll be
covering in this week.

video: welcome to week 1 (1:06)

\subsection{Weekly learning outcomes}\label{subsec:weekly-learning-outcomes}
* Describe the basic properties of multivariate normal distribution.
* Diagnose deviations from normality in real-world datasets.
* Obtain and plot point estimates and confidence intervals for vector means and
differences of vector means.
* Test linear hypotheses about vector means and differences of vector means.
* Explain the assumptions underlying the above inferential procedures.
* Use R packages to perform analyses.

\subsection{Topics we will cover are:}\label{subsec:topics-we-will-cover-are:}
* Topic 1: Exploratory data analysis of multivariate data
* Topic 2: The multivariate normal distribution
* Topic 3: Estimation of vector mean and of variance matrices: point estimates
* Topic 4: Confidence intervals and hypothesis tests for the mean vector

\subsection{Optional readings}\label{subsec:optional-readings}
An alternative presentation of the concepts for this week can be found in:

Johnson, R. A., \& Wichern, D. (2008). Applied Multivariate Statistical Analysis
(6th ed.). Pearson Prentice Hall.
* 4.1–4.2, 4.6
* 5.1–5.5 and 6.1–6.3

Härdle, W. K., \& Simar, L. (2014). Applied Multivariate Statistical Analysis
(4th ed.). Springer.
* 1.4–1.7
* 3.1–3.3
* 4.4, 5.1–5.3, 6.1
* 7.1–7.2

All readings are available from the course Leganto reading list. Please keep in
mind that you will need to be logged into Moodle to access the Leganto reading
list.

\section{Questions about this week's topics?}\label{sec:questions-about-this-week's-topics?}
This week's topics were prepared by Dr P. Krivitsky.
If you have any questions or comments, please post them under Discussion or
email directly:
p.krivitsky@unsw.edu.au

\section{Exploratory data analysis of multivariate data}\label{sec:exploratory-data-analysis-of-multivariate-data}

\subsection{Introduction}\label{subsec:introduction}
We begin by taking a look at how to summarise multivariate data—with a focus on
quantitative data—and to visualise it.

\section{Data organisation}\label{sec:data-organisation}
Assume, we are dealing with $p \geq 1$ *variables*.
The values of these variables are all recorded for each distinct item,
individual, or experimental trial.
Each of these three words will sometimes be substituted by the word
\("\)case\("\).
We will use the notation $x_{ij}$ to indicate a particular value of the
$i^{th}$ variable that is observed on the $j^{th}$ case.
Consequently, $n$ measurements on $p$ variables can be represented in a form
of a matrix

\[
X =
\begin{pmatrix}
x_{11} & x_{12} & \dots & x_{1j} & \dots & x_{1n} \\
x_{21} & x_{22} & \dots & x_{2j} & \dots & x_{2n} \\
x_{i1} & x_{i2} & \dots & x_{ij} & \dots & x_{in} \\
x_{p1} & x_{p2} & \dots & x_{pj} & \dots & x_{pn} \\
\end{pmatrix}
\]

The matrix $X$ above contains the data consisting of all the observations on
all the variables.
This way of representing the data allows easy manipulations
to be performed in order to obtain some easy descriptive statistics for each
of the variables.

\section{Basic summaries}\label{sec:basic-summaries}
For example, the sample mean of the second variable is just $\bar{x}_2 = \dfrac{1}{n} \sum_{j=1}^{n} x_{2j}$,
the sample variance of the second variable is just $s_2^2 = \dfrac{1}{n}\sum_{j=1}^{n} x_{2j}$
(Note that for the sample variance we shall sometimes use the divisor of $n - 1$
rather than $n$ and each time this will be differentiated by displaying the
appropriate expression).

The sample covariance (the simple measure of linear association between
variables 1 and 2) is given by $s_{12} = \dfrac{1}{n} \sum_{j=1}^{n}(x_{1j} - \bar{x}_1)(x_{2j} - \bar{x}_2)$
and one can understand easily how $s_{ik}, i = 1,2, \dots, p, k = 1,2,\dots,p$ can be
defined.
Finally, the sample correlation coefficient (the measure of linear
association between two variables that does not depend on the units of
measurement) can be defined.
The sample correlation coefficient of the $i^{th}$
and $k^{th}$ variables is defined by $r_{ik} = \dfrac{s_{ik}}{\sqrt{s_{ii}\sqrt{s_{kk}}}}$.
Because of the well-known
Cauchy–Bunyakovsky–Schwartz Inequality, $|r_{ik}| \leq 1$ holds.
Note also that $r_{ik}$ for all $i = 1,2,\dots,p$ and $k=1,2,\dots,p$ holds.

It should be repeatedly noted that the sample correlations and covariance are
useful only when trying to measure the linear association between two variables.
Their value is less informative and is misleading in cases of nonlinear association.
In this case one needs to invoke the correlation quotient instead.

Since covariance and correlation coefficients are routinely calculated and
analysed they are very widely used and provide useful numerical summaries of
association when the data do not exhibit obvious nonlinear patterns of
association.

The descriptive statistics that we discussed until now are usually organised
into arrays, namely:
* Vector of sample means $\bar{x} = \big( \bar{x}_1, \bar{x}_2, \bar{x}_p \big)^T$
* Matrix of sample variances and covariances
\[
\begin{matrix}
\end{matrix}
\]
* Matrix of sample correlations  $$

\section{Visualisation}\label{sec:visualisation}
\subsection{Initial graphical representation of the data}\label{subsec:initial-graphical-representation-of-the-data}
some simple characteristics of the data are worth studying before the actual
multivariate analysis would begin:
* drawing a scatter plot of the data
* calculating simple univariate descriptive statistics for each variable
* calculating sample correlation and covariance coefficients
* linking multiple two-dimensional scatterplots

# Example: R activity on basic multivariate summaries and visualisation
The first practice example will demonstrate multivariate summary statistics and graphics.
You can use the RStudio console here or complete the exercise on your own RStudio

insert video

To complete this task select the \('\)Multivariate\_exploration\_visualisation.Rmd'
in the \('\)Files' section of RStudio.
Follow the example contained within the RMD file.

If you choose to complete the example in your own RStudio, upload the following
file: `multivariate\_exploration\_visualisation.rmd`

The output of the RMD file is also displayed below:

In R there are implemented in the `base::colMeans`, `stats::cor`, `graphics::plot`,
`graphis::pairs`, `GGally:ggpairs`.
Here the format is `PACKAGE::FUNCTION` and you can learn more by running

`library(PACKAGE)
FUNCTION(PACKAGE)`

Next, work through the basic multivariate summaries and visualisation example
in the next section.

video: R tutorial on Basic Multivariate summaries and visualisations

Example: R activity on basic multivariate summaries and visualisation

The output of the RMD file is also displayed below:
\end{document}