%! Author = noone
%! Date = 22/10/22

% Preamble
\documentclass[11pt]{article}

% Packages
\usepackage{amsmath}
\usepackage{graphicx}

% Document
\begin{document}







\begin{figure}
    \centering
    \includegraphics{/home/noone/Documents/GitHub/multivariate_statistics/o_week/resources/r_logo}
    \caption{}
    \label{fig:}
\end{figure}

RStudio is an integrated development environment (IDE) for R. It includes a
console, syntax-highlighting editor that supports direct code execution, as
well as tools for plotting, history, debugging and workspace management.
R and RStudio are open-source and free to use.

You should set up your R environment as soon as possible.
You need to be ready to use R straight from the start of the course and will be
required to perform a basic data and regression analysis using R in the first
assessment in Week 1.
Rmarkdown is a markup language, which is an excellent tool for report writing
with R.

Installing R and RStudio

You can download RStudio suitable for your system for free from its official
website.
The Ed Platform in which you will be completing most of your work also has an
RStudio emulator and you can use the emulator to complete the R demonstrations
and challenges throughout the course.

There are numerous online video tutorials devoted to R and RStudio installation.
You might like to start here:

Using R

As an introduction to using R, watch the following video tutorial on some basic
arithmetic in R:

There are some many videos in this series from MarinStatsLectures you are
welcomed to use to supplement the materials provided in this course.

R for Data Science: https://r4ds.had.co.nz/

For those who are new to R, you will find the book R for Data Science by Garrett
Grolemund and Hadley Wickham a useful introduction on how to do data science
using R.
R for Data Science will show you how to get your data into R, get it into the
most useful structure, transform it, visualise it, model it, and communicate it.
The online version of the book is free to use and is licensed under the
Creative Commons Attribution-NonCommercial-NoDerivs 3.0 License
(CC BY-NC-ND 3.0 US).

Also, this video series from LinkedIn learning is a great reference for those
new to R. You will need your UNSW details to log in and view the videos:
https://www.linkedin.com/learning/learning-r-2/r-for-data-science?u=2087740

Rmarkdown

Rmarkdown is a markup language, which is an excellent tool for report writing
with R.

how to author RMarkdown reports

You may like to consider completing this free online short course from DataCamp
for an introduction to Rmarkdown.

https://www.datacamp.com/courses/reporting-with-r-markdown

Basic commands

Use the below R reference guide for a short overview of the most important
functions you need to be familiar with from the start of this course.

\end{document}